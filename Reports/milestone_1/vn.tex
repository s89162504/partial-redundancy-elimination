\chapter{Value Numbering} \label{ch:vn}

\section{Approach}
Prior research has shown that value numbering can increase opportunities for PRE.
LLVM presently has a GVN-PRE pass which exploits this. However, value numbering in 
GVN-PRE is tightly coupled with the code for removing redundancies, and hence
we deem it not useful for our purposes. We have written our own value numbering
pass which feeds expression value numbers to the PRE stage. It should be noted,
     however, that we do not implement value numbering from scratch and use an
     old (now defunct) LLVM pass as a starting point. Most importantly, we
     augment the basic value numbering in the following ways - 
     
\begin{itemize}     
  \item Add the notion of leader expression (described below), with associated
  data structures and functions. 
  \item Functionality to support value-number-based bitvectors rather than
  expression-name-based bitvectors. 
  \item (Optimization 1) If the expression operator is either of AND, OR, CMP::EQ
  or CMP::NE, and the operands have the same value number, we replace all uses
  accordingly and then delete the expression.
  \item (Optimization 2) If all operands of an expression are constants, then we can 
  evaluate and propagate constants. 
  \item (Optimization 3) If one operand of an expression is a constant (0 or 1), then 
  we can simplify the expression. e.g. {a+0 = a} , {b*1 = b}.
  \item (Optimization 4) If the incoming expressions to a PHI node have the same value 
  number, then the PHI node gets that same value number
\end{itemize}  
  As per our testing, optimizations 2 and 3 are also done by the reassociation
  pass in LLVM. In our final code we omit our implementation and rely on the
  more robust LLMV pass. Optimizations 1 and 4, however, are still our contribution.

\section{Notion Of Leader Expression}  
The value numbering algorithm computes the RPO solution as outlined in Tarjan's
SCC paper\cite{Cooper95scc-basedvalue}. It goes over the basic blocks in
reverse post order and adds new expressions to a hash table based on the
already computed value numbers of the operands. We call an expression a
`leader' if at the time of computing its value number, the value number doesn't
already exist in the hash table. In other words, out of a potentially large set of
expressions that map to a particular value number, the leader expression was
the first to be encountered while traversing the function in reverse post
order. Leader expressions are vital to our algorithm as outlined in the section
on PRE.
